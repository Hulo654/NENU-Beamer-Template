%----------------------------------------------------------------
% 本主题基于 beamer 内置主题 Berlin 修改而来。
% 底部导航栏的设置参考了 https://www.guanjihuan.com/archives/2662 ,感谢。
%-----------------------------------------------------------------
\documentclass[10pt,aspectratio=43]{beamer} % 其中 aspectratio=43 是设置屏幕比例,也可改为 169.
%\usetheme[compress]{Berlin}
\usetheme{default}
\usefonttheme[onlymath]{serif} % 字体主题
\usecolortheme{whale} % 颜色主题
\useinnertheme{rectangles} % 内部主题
\useoutertheme[subsection=false]{miniframes} % 外部主题

\setbeamercolor{section in head/foot}{fg=white, bg=black} % 设置顶部导航栏颜色
\setbeamercolor{author in head/foot}{fg=white, bg=black} % 设置底部作者区域颜色
\setbeamercolor{institute in head/foot}{fg=white, bg=gray} % 设置底部单位区域颜色
\setbeamercolor{date in head/foot}{fg=white, bg=black} % 设置底部日期区域颜色
\setbeamertemplate{navigation symbols}{} % 删除导航按钮

% 在导言区加上这一片,在底部可以显示作者、标题、日期和页数。
% 本人将其中标题修改为了机构/单位。
\setbeamertemplate{footline}{%
	\leavevmode%
	\hbox{%
		\begin{beamercolorbox}[wd=.333333\paperwidth,ht=2.65ex,dp=1ex,center]{author in head/foot}%
			\usebeamerfont{author in head/foot}\insertshortauthor
		\end{beamercolorbox}%
		\begin{beamercolorbox}[wd=.333333\paperwidth,ht=2.65ex,dp=1ex,center]{institute in head/foot}%
			\usebeamerfont{institute in head/foot}\insertshortinstitute
		\end{beamercolorbox}%
		\begin{beamercolorbox}[wd=.333333\paperwidth,ht=2.65ex,dp=1ex,right]{date in head/foot}%
			\usebeamerfont{date in head/foot}\insertshortdate{}\hspace*{2em}
			\insertframenumber{} / \inserttotalframenumber\hspace*{2ex}
	\end{beamercolorbox}}%
	\vskip0pt%
}

% 右下角显示页码,会破坏底部导航栏,故不用。
%\setbeamertemplate{footline}[frame number]{}

% 调用一些包
\usepackage{ctex, hyperref}
\usepackage[T1]{fontenc}
\usepackage{latexsym,amsmath,xcolor,multicol,booktabs,calligra}
\usepackage{graphicx,pstricks,listings,stackengine}

\title{东北师范大学学术汇报}
\author{Sun Kaixin}
\institute{Northeast Normal University}
\date{Sept 22, 2024}

\begin{document}
	% 创建标题页
	\begin{frame}
		\maketitle
	\end{frame}
%-----------------------------------------------------------
	% 创建目录页
	\begin{frame}{Outline} % or Contents
		\tableofcontents
	\end{frame}
%-----------------------------------------------------------
	\section{第一部分}
	\begin{frame}
		\frametitle{Frame Title 1}
		% 无序列表
		\begin{itemize} 
			\item first \\~\\ % 注意“\\~\\”符号作用是增加段落之间的距离,本质上是增加了一个空行。
			\item second \\~\\
			\item third \\~\\
		\end{itemize}
	\end{frame}
%-----------------------------------------------------------
	\begin{frame}
		\frametitle{Frame Title 2}
		\begin{itemize} 
			% change color
			\item {\color{blue} This is a blue block.} \\~\\
			\item {\color{red} And this is a red one.} \\~\\
			\item third \\~\\
		\end{itemize}
	\end{frame}
%-----------------------------------------------------------
	\begin{frame}
		\frametitle{Frame Title 3}
		\begin{itemize} 
			% \textbf {要加粗的文本}
			\item 外边,\textbf {傍晚的斜阳}正照在场子上,使得那一簇簇山茱萸的白花在一片娇绿的背景上烘托得分外鲜明。\\~\\
			\item 那哥儿俩骑来的两匹红毛马儿,现在夹道里吊着。马脚跟前有一群到处随行的猎犬在那里吵架。\\~\\
			\item 三参数 Logistic 模型:
			{\LARGE $$
				P_{i}=c_{i} +\tfrac{1-c_{i}}{1+e^{[-Da_{i}(\theta -b_{i})]}}
				$$} % 插入数学公式。
		\end{itemize}
	\end{frame}
%-----------------------------------------------------------
	\section{第二部分}
	\begin{frame}
		\frametitle{Frame Title 4}
		\begin{itemize} 
			\item “Wonderful,” \textbf {Morgan} said. Then she reached into her robe and pulled out a scroll. \\~\\
			\item “You’ve solved two riddles so far,” she said. “Here is your third.” She handed the scroll to Annie. \\~\\
			\item  “And for your research—” She pulled a book out from her robe and handed it to Jack. \\~\\
		\end{itemize}
	\end{frame}
%-----------------------------------------------------------
	\begin{frame}
		\frametitle{Frame Title 5}
		\begin{itemize} 
			\item first \\~\\
			\item second \\~\\
			\item third \\~\\
		\end{itemize}
	\end{frame}
%-----------------------------------------------------------
	\begin{frame}
		\frametitle{Frame Title 6}
		\begin{itemize} 
			\item first \\~\\
			\item second \\~\\
			\item third \\~\\
		\end{itemize}
	\end{frame}
%-----------------------------------------------------------
	\section{第三部分}
	\begin{frame}
		\frametitle{Frame Title 7}
		\begin{itemize} 
			\item first \\~\\
			\item second \\~\\
			\item third \\~\\
		\end{itemize}
	\end{frame}
%-----------------------------------------------------------
	\begin{frame}
		\frametitle{Frame Title 8}
		\begin{itemize} 
			\item first \\~\\
			\item second \\~\\
			\item third \\~\\
		\end{itemize}
	\end{frame}
%-----------------------------------------------------------
	\begin{frame}
		\frametitle{Frame Title 9}
		\begin{itemize} 
			\item first \\~\\
			\item second \\~\\
			\item third \\~\\
		\end{itemize}
	\end{frame}
%-----------------------------------------------------------
	\section{第四部分}
	\begin{frame}
		\frametitle{Frame Title 10}
		\begin{itemize} 
			\item first \\~\\
			\item second \\~\\
			\item third \\~\\
		\end{itemize}
	\end{frame}
%-----------------------------------------------------------
	\begin{frame}
		\frametitle{Frame Title 11}
		\begin{itemize} 
			\item first \\~\\
			\item second \\~\\
			\item third \\~\\
		\end{itemize}
	\end{frame}
%-----------------------------------------------------------
	\begin{frame}
		\frametitle{Frame Title 12}
		\begin{itemize}
			\item first \\~\\
			\item second \\~\\
			\item third \\~\\
		\end{itemize}
	\end{frame}
%-----------------------------------------------------------
	\begin{frame}
		\frametitle{{}} % 结尾页 frametitle 输入 "{}" 不显示内容但保留背景框。
		\begin{center} % 居中
		{\Huge Thanks!} % 最大
	    \end{center}
	\end{frame}
%-----------------------------------------------------------
\end{document}
